\section{Zadanie 1}
Sprawdzaj�c komunikacj� ze stanowiskiem skorzystali�my z dw�ch funkcji zapewnionych przez prowadz�cego \verb+MinimalWorkingExample.m+ raz \verb+sendControlsToG1AndDisturbance.m+. Pierwsza z nich pozwala w prosty spos�b, konfiguruj�c port, na kt�rym odbywa si� komunikacja, zadawa� warto�ci sterowania na poszczeg�lne elementy wykonawcze stanowiska.
\begin{lstlisting}
        sendControls(1, 50);
\end{lstlisting}
Zadaj�c warto�� 0 i 50 na wentylator widzimy i s�yszymy czy komunikacja zachodzi.

Sterowanie grza�k� w tym zadaniu odbywa�o si� z u�yciem drugiej z wymienionych funkcji, aby zrealizowa� zadane w poleceniu zak��cenia. Funkcja \verb+sendControlsToG1AndDisturbance.m+ przyjmuje dwa argumenty: warto�� sterowania grza�k� \verb+G1+ i zak��cenia \verb+Z+.
\begin{lstlisting}
        sendControlsToG1AndDisturbance(35, Z);
\end{lstlisting}

Kolejnym krokiem by�o okre�lenie warto�ci temperatury w punkcie pracy: \verb+G1+ = 35, \verb+W1+ = 50, \verb+Z+ = 0. Dla takich nastaw temperatura wynosi�a ok. $32^{\mathrm{o}}{\mathrm{C}}$.