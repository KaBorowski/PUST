\section{Zadanie 1}
W celu sprawdzenia poprawno�ci warto�ci sygna��w w punkcie pracy pobudzili�my obiekt sygna�em o sta�ej warto�ci r�wnej $U_\mathrm{pp} = 0$, przy sta�ym zak��ceniu $Z_\mathrm{pp} = 0$. Spodziewana warto�� wyj�cia to $Y_\mathrm{pp} = 0$.

Zadanie wykonali�my przy u�yciu skryptu \verb+zad1.m+, kt�ry symuluje badan� sytuacj�. Przy opisanym wy�ej pobudzeniu obiekt, zgodnie z oczekiwaniami, stabilizuje si� w $Y_\mathrm{pp} = 0$ ($Rys.\ 3.1$).
\begin{figure}[H]
	\centering
	\begin{tikzpicture}
	\begin{axis}[
	width=5.667in,
	height=1.645in,
	xmin=0,xmax=300,ymin=-1,ymax=1,
	xlabel={$k$},
	ylabel={$U(k)$},
	legend pos=south east,
	y tick label style={/pgf/number format/1000 sep=},
	]
	\addplot[const plot,blue] file {rysunki/data/Zad1/zad1_u.csv};
	\end{axis}
	\end{tikzpicture}
	\begin{tikzpicture}
	\begin{axis}[
	width=5.667in,
	height=1.645in,
	xmin=0,xmax=300,ymin=-1,ymax=1,
	xlabel={$k$},
	ylabel={$Z(k)$},
	legend pos=south east,
	y tick label style={/pgf/number format/1000 sep=},
	]
	\addplot[const plot,blue] file {rysunki/data/Zad1/zad1_z.csv};
	\end{axis}
	\end{tikzpicture}
	\begin{tikzpicture}
	\begin{axis}[
	width=5.667in,
	height=1.645in,
	xmin=0,xmax=300,ymin=-1,ymax=1,
	xlabel={$k$},
	ylabel={$Y(k)$},
	legend pos=south east,
	y tick label style={/pgf/number format/1000 sep=},
	]
	\addplot[const plot,blue] file {rysunki/data/Zad1/zad1_y.csv};
	\end{axis}
	\end{tikzpicture}
	\caption{Odpowied� w punkcie pracy}
	\label{r_pgfplots_funkcje}
\end{figure}
