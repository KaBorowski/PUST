\chapter{Projekt}
Realizuj�c projekt, zasymulowali�my obiekt korzystaj�c z podanej przez prowadz�cego funkcji \verb+symulacja_obiektu10Y+. Zgodnie z wymaganiami, u�yli�my jej aby uzyska� obiekt o nast�puj�cym r�wnaniu:


\begin{center}
\verb+Y(k)=symulacja_obiektu10Y(U(k-10),U(k-11),Y(k-1),Y(k-2))+
\end{center}


U�yta funkcja umo�liwia wyznaczenie sygna�u wyj�ciowego procesu $(Y)$, w aktualnej chwili dyskretnej $k$, w zale�no�ci od warto�ci sygna�u wej�ciowego $(U)$ i wyj�ciowego w poprzednich chwilach pr�bkowania.

\section{Zadanie 1}
Celem pierwszego zadania by�o sprawdzenie poprawno�ci, podanego w projekcie, punktu pracy obiektu: $U_pp = 3$, $Y_pp = 0,9$. Aby to sprawdzi�, wykonali�my skok sterowania do warto�� 3 i zaobserwowali�my na jakim poziomie ustabilizuje wyj�cie $Y$.